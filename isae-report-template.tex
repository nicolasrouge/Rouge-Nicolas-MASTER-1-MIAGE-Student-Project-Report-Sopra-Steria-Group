% A LaTeX (non-official) template for ISAE projects reports
% Copyright (C) 2014 Damien Roque
% Version: 0.2
% Author: Damien Roque <damien.roque_AT_isae.fr>

\documentclass[a4paper,12pt]{book}
\usepackage[utf8]{inputenc}
\usepackage[T1]{fontenc}
\usepackage{float}
\usepackage[frenchb]{babel} % If you write in French
%\usepackage[english]{babel} % If you write in English
\usepackage{a4wide}
\usepackage{graphicx}
\graphicspath{{images/}}
\usepackage{subfig}
\usepackage{tikz}
\usetikzlibrary{shapes,arrows}
\usepackage{pgfplots}
\pgfplotsset{compat=newest}
\pgfplotsset{plot coordinates/math parser=false}
\newlength\figureheight
\newlength\figurewidth
\pgfkeys{/pgf/number format/.cd,
set decimal separator={,\!},
1000 sep={\,},
}
\usepackage{ifthen}
\usepackage{ifpdf}
\ifpdf
\usepackage[pdftex]{hyperref}
\else
\usepackage{hyperref}
\fi
\usepackage{color}
\hypersetup{%
colorlinks=true,
linkcolor=black,
citecolor=black,
urlcolor=black}

\renewcommand{\baselinestretch}{1.05}
\usepackage{fancyhdr}
\pagestyle{fancy}
\fancyfoot{}
\fancyhead[LE,RO]{\bfseries\thepage}
\fancyhead[RE]{\bfseries\nouppercase{\leftmark}}
\fancyhead[LO]{\bfseries\nouppercase{\rightmark}}
\setlength{\headheight}{15pt}

\let\headruleORIG\headrule
\renewcommand{\headrule}{\color{black} \headruleORIG}
\renewcommand{\headrulewidth}{1.0pt}
\usepackage{colortbl}
\arrayrulecolor{black}

\fancypagestyle{plain}{
  \fancyhead{}
  \fancyfoot[C]{\thepage}
  \renewcommand{\headrulewidth}{0pt}
}

\makeatletter
\def\@textbottom{\vskip \z@ \@plus 1pt}
\let\@texttop\relax
\makeatother

\makeatletter
\def\cleardoublepage{\clearpage\if@twoside \ifodd\c@page\else%
  \hbox{}%
  \thispagestyle{empty}%
  \newpage%
  \if@twocolumn\hbox{}\newpage\fi\fi\fi}
\makeatother

\usepackage{amsthm}
\usepackage{amssymb,amsmath,bbm}
\usepackage{array}
\usepackage{bm}
\usepackage{multirow}
\usepackage[footnote]{acronym}

\newcommand*{\SET}[1]  {\ensuremath{\mathbf{#1}}}
\newcommand*{\VEC}[1]  {\ensuremath{\boldsymbol{#1}}}
\newcommand*{\FAM}[1]  {\ensuremath{\boldsymbol{#1}}}
\newcommand*{\MAT}[1]  {\ensuremath{\boldsymbol{#1}}}
\newcommand*{\OP}[1]  {\ensuremath{\mathrm{#1}}}
\newcommand*{\NORM}[1]  {\ensuremath{\left\|#1\right\|}}
\newcommand*{\DPR}[2]  {\ensuremath{\left \langle #1,#2 \right \rangle}}
\newcommand*{\calbf}[1]  {\ensuremath{\boldsymbol{\mathcal{#1}}}}
\newcommand*{\shift}[1]  {\ensuremath{\boldsymbol{#1}}}

\newcommand{\eqdef}{\stackrel{\mathrm{def}}{=}}
\newcommand{\argmax}{\operatornamewithlimits{argmax}}
\newcommand{\argmin}{\operatornamewithlimits{argmin}}
\newcommand{\ud}{\, \mathrm{d}}
\newcommand{\vect}{\text{Vect}}
\newcommand{\sinc}{\ensuremath{\mathrm{sinc}}}
\newcommand{\esp}{\ensuremath{\mathbb{E}}}
\newcommand{\hilbert}{\ensuremath{\mathcal{H}}}
\newcommand{\fourier}{\ensuremath{\mathcal{F}}}
\newcommand{\sgn}{\text{sgn}}
\newcommand{\intTT}{\int_{-T}^{T}}
\newcommand{\intT}{\int_{-\frac{T}{2}}^{\frac{T}{2}}}
\newcommand{\intinf}{\int_{-\infty}^{+\infty}}
\newcommand{\Sh}{\ensuremath{\boldsymbol{S}}}
\newcommand{\C}{\SET{C}}
\newcommand{\R}{\SET{R}}
\newcommand{\Z}{\SET{Z}}
\newcommand{\N}{\SET{N}}
\newcommand{\K}{\SET{K}}
\newcommand{\reel}{\mathcal{R}}
\newcommand{\imag}{\mathcal{I}}
\newcommand{\cmnr}{c_{m,n}^\reel}
\newcommand{\cmni}{c_{m,n}^\imag}
\newcommand{\cnr}{c_{n}^\reel}
\newcommand{\cni}{c_{n}^\imag}
\newcommand{\tproto}{g}
\newcommand{\rproto}{\check{g}}
\newcommand{\LR}{\mathcal{L}_2(\SET{R})}
\newcommand{\LZ}{\ell_2(\SET{Z})}
\newcommand{\LZI}[1]{\ell_2(\SET{#1})}
\newcommand{\LZZ}{\ell_2(\SET{Z}^2)}
\newcommand{\diag}{\operatorname{diag}}
\newcommand{\noise}{z}
\newcommand{\Noise}{Z}
\newcommand{\filtnoise}{\zeta}
\newcommand{\tp}{g}
\newcommand{\rp}{\check{g}}
\newcommand{\TP}{G}
\newcommand{\RP}{\check{G}}
\newcommand{\dmin}{d_{\mathrm{min}}}
\newcommand{\Dmin}{D_{\mathrm{min}}}
\newcommand{\Image}{\ensuremath{\text{Im}}}
\newcommand{\Span}{\ensuremath{\text{Span}}}

\newtheoremstyle{break}
  {11pt}{11pt}%
  {\itshape}{}%
  {\bfseries}{}%
  {\newline}{}%
\theoremstyle{break}

%\theoremstyle{definition}
\newtheorem{definition}{Définition}[chapter]

%\theoremstyle{definition}
\newtheorem{theoreme}{Théorème}[chapter]

%\theoremstyle{remark}
\newtheorem{remarque}{Remarque}[chapter]

%\theoremstyle{plain}
\newtheorem{propriete}{Propriété}[chapter]
\newtheorem{exemple}{Exemple}[chapter]

\parskip=5pt
%\sloppy

\begin{document}

%%%%%%%%%%%%%%%%%%
%%% First page %%%
%%%%%%%%%%%%%%%%%%

\begin{titlepage}
\begin{center}

\includegraphics[width=0.6\textwidth]{logo_parisNanterre.png}\\[1cm]
{\large Master MIAGE}\\[0.5cm]
{\large Méthodes Informatiques Appliquées à la Gestion des Entreprises}\\[0.5cm]
{\large Promotion M1 Classique}\\[0.5cm]
{\large Année 2017-2018}\\[0.5cm]
%{\large Type de projet}\\[0.5cm]

% Title
\rule{\linewidth}{0.5mm} \\[0.4cm]
{ \huge \bfseries  Développeur au sein du portail de pilotage PPIL\\[0.4cm] }
\rule{\linewidth}{0.5mm} \\[1.5cm]
{\large Client : CNAM (Caisse Nationale de l'Assurance Maladie)}\\[0.5cm]
\includegraphics[width=0.8\textwidth]{logo-soprasteria-2.png}\\[1cm]
% Author and supervisor
\noindent
\begin{minipage}{0.4\textwidth}
  \begin{flushleft} \large
    \emph{Auteur :}\\
    M. Nicolas \textsc{Rouge}\\
  \end{flushleft}
\end{minipage}%
\begin{minipage}{0.6\textwidth}
  \begin{flushright} \large
    \emph{Tuteur de l'entreprise :} \\
    M.~Driss \textsc{IHLALI}\\
    \emph{Tuteur de l'ecole :} \\
    M.~Fabrice \textsc{LEGOND-AUBRY}
  \end{flushright}
\end{minipage}

\vfill

% Bottom of the page
{\large Version 1.3 du\\ \today}

\end{center}
\end{titlepage}

%%%%%%%%%%%%%%%%%%%%%%%%%%%%%
%%% Non-significant pages %%%
%%%%%%%%%%%%%%%%%%%%%%%%%%%%%

\frontmatter

\chapter*{Remerciements}
Je tiens à remercier toute l'équipe PPIL, avec qui j'ai travaillé au quotidien. Toute l'équipe a su m'accueillir et me donner de bons conseils, toujours avec bienveillance. 
Achref a été mon développeur référent. Il a su me donner de bons conseils et suivre ma progression au sein de mon stage. et lors de mes différentes missions de développement.
Je remercie aussi Driss pour son accueil et sa qualité d'écoute.

De nombreuses personnes ont contribué à ma bonne intégration chez Sopra Steria. Grâce notament aux réunions hebdomadaires de stagiaires. Merci à Thomas Richard pour sa présence aux réunions stagiaire, ainsi qu’à tous les stagiaires travaillant pour la CNAM au sein du pôle de Montreuil et à l'Azip.
Je remercie aussi Sara pour le suivi de mon parcours et mon intégration au sein du groupe.
Et enfin merci à toute l'équipe accueil qui a contribué à créer du dynamisme au sein du pôle.

%PPIL : Driss, Nicolas, Arkaia, Huifong, Sonia, Julie, Thomas et Marwan
%
\clearpage
\tableofcontents

\clearpage
\listoffigures

\clearpage
\chapter*{Liste des sigles et acronymes}
\begin{acronym}[CP-OFDMX] % Give the longest acronym here
\acro{BA}{\emph{Business Analyst}}
\acro{BDD}{\emph{Base De Données}}
\acro{BU}{\emph{Business Unit}}
\acro{CNAM}{\emph{Caisse Nationale d'Assurance Maladie}}
\acro{CP}{\emph{Chef de Projet}}
\acro{MEP}{\emph{Mise en producion}}
\acro{MOE}{\emph{Maitrise d’Œuvre }}
\acro{MOA}{\emph{Maitrise d’Ouvrage}}
\acro{MPD}{\emph{Modèle Physique de données}}
\acro{SFG}{\emph{Spécifications Fonctionnelles Générales}}
\acro{SFD}{\emph{ Spécifications Fonctionnelles Détaillées}}
\acro{SSG}{\emph{Sopra Steria Group}}
\acro{STD}{\emph{Spécifications Techniques Détaillées}}
\acro{V1}{\emph{Réunion hebdomadaire}}
\acro{V2}{\emph{Réunion mensuelle}}
\acro{PM}{\emph{Project Manager}}
\acro{PMO}{\emph{Project Management Office}}
\acro{SB}{\emph{Solution Builder}}
\end{acronym}

%%%%%%%%%%%%%%%%%%%%%%%%%%%%%%%%%%%%%%%%%%%%
%%% Content of the report and references %%%
%%%%%%%%%%%%%%%%%%%%%%%%%%%%%%%%%%%%%%%%%%%%

\mainmatter
\pagestyle{fancy}

\cleardoublepage

\chapter*{Introduction}
\addcontentsline{toc}{chapter}{Introduction}
\markboth{Introduction}{Introduction}
\label{chap:introduction}
%\minitoc

Sopra Steria Group m'a accueilli pour mon stage que j'ai effectué dans le cadre de ma formation MASTER 1 MIAGE. Je tenais à effectuer mon stage dans une entreprise novatrice et orientée vers le digital. Ce stage a été pour moi l'occasion d'intégrer la vie active dans le domaine qui me motive. Cela a été ma première expérience de développeur en entreprise.

Lors de ma recherche de stage, je me suis naturellement tourné vers Sopra Steria, entreprise de service numérique qui est un des leader européen de la transformation numérique.
J'ai postulé chez Sopra Steria pour la diversité des projets et des secteurs d'activités ainsi que le grand nombre de métiers. Différentes qualités que l'on retrouve chez les ESN. 

Sopra Steria Group est née de la fusion de deux SSII : Sopra et Steria en Janvier 2015.
L'entreprise intervient sur plusieurs secteurs, chaque secteur est constitué de plusieurs agences qui fonctionnent comme des entreprises indépendantes. 

C'est à la suite d’un entretien que j'ai été recruté au sein de l’une des 4 agences s'occupant du secteur public : l’agence 151 qui s’occupe du domaine "Santé, Social, Emploi".
Chaque agence est divisée en plusieurs équipes qui gèrent différents projets.

Lors de ce stage, mon objectif était de découvrir le métier de développeur au sein d'une entreprise. Je suis heureux d'avoir pu intégrer une équipe. j'ai pu découvrir tous les aspects de gestion d'un projet, avec les différentes étapes, procédures et des différents rôles que chacun occupe.

J'ai intégré l'équipe PPIL qui développe un outil pour la CNAM. L'outil existe depuis 2010. Il a pour objectif, de permettre en fonction de leurs profils (par exemple Manager ou Chef de projet) d'effectuer le reporting et le suivi des différents projets. 

Il est primordial de préciser que le client du projet est la CNAM, un des plus gros organisme du secteur public français. L'organisme est en pleine révolution digitale. Sopra Steria Group et ses collaborateurs l’accompagnent sur le chemin de la modernisation. C’est dans ce contexte que je suis intervenu sur PPIL. 

Travailler pour un client du service public a un intérêt particulier pour moi car les actions menées au quotidien par Sopra Steria ont un impact direct sur la vie de nos concitoyens.

Je suis aussi intervenu sur un projet "stagiaires" interne à Sopra Steria Pôle métier agence 151 : Gestion des compétences des différents intervenants du plateau. L'objectif principal de l'outil est de permettre aux chefs de projets ou administrateurs de projets de construire des équipes en fonction des compétences techniques des collaborateurs.

%%% Local Variables: 
%%% mode: latex
%%% TeX-master: "isae-report-template"
%%% End: 

\chapter{Contexte / Environnement}
\label{chap:premierchapitre}

Dans ce chapitre, nous allons présenter l'entreprise dans laquelle j'ai travaillé, Sopra Steria, sa structure, l'agence 151 du secteur public, ainsi que le client auquel nos développement sont destinés : la Cnam.

\section{Sopra Steria}

Sopra Steria est né de la fusion en 2014 de deux des plus anciennes Entreprises de Services du Numérique françaises, Sopra et Steria, fondées respectivement en 1968 et 1969 et marquées toutes deux par un fort esprit entrepreneurial ainsi qu'un grand sens de l’engagement collectif au service de ses clients.

Le Groupe s'affirme comme un des leaders européens de la transformation numérique.

Suite à cette fusion le groupe compte près de 42 000 collaborateurs répartis dans plus de 20 pays et a réalisé un chiffre d’affaire de 3,8 milliards d’euros en 2017.

\begin{figure}[!h]
\centering
\includegraphics[width=0.8\textwidth]{images/chiffres_cles2017.jpg}
\caption{Sopra Steria : leader européen de la transformation numérique}
\end{figure}


Sopra Steria devient alors l'un des leader Européen de la transformation numérique.

\begin{figure}[!h]
\centering
\includegraphics[width=0.5\textwidth]{images/payssoprasteria.png}
\caption{Sopra Steria : Répartition de l'activité en fonction des pays}
\end{figure}

Cette fusion a pris grâce à une forte complémentarité entre les deux entreprises : Sopra Group étant très implanté en France et peu à l'international, et Steria étant une entreprise reconnue à l'international, notamment en Europe.

//L’entreprise intervient dans de nombreux secteurs et domaines d’activité, et a du apprendre à faire face aux problèmes de managements intrinsèquement liés à sa taille et polyvalence.//

\begin{figure}[!h]
\centering
\includegraphics[width=0.5\textwidth]{images/metier_soprasteria.png}
\caption{Sopra Steria : Répartition des activités en fonction des métiers}
\end{figure}

Nous allons expliquer plus en détail les choix organisationnels de l’entreprise en prenant comme exemple notre secteur d’activité.

\section{Organisation du groupe}

//L'entreprise a choisi de diviser les secteurs d’activité et limiter le nombre d’échelons hiérarchiques au sein de chaque secteur, l’entreprise a également donné un grand pouvoir décisionnel et une grande indépendance aux collaborateurs exerçants des fonctions à responsabilités.//

La figure ci-dessous illustre la première division s’effectuant donc par secteur d’activité :

\begin{figure}[!h]
\centering
\includegraphics[width=0.8\textwidth]{images/secteurActivite.png}
\caption{Sopra Steria : Secteurs d'activités}
\end{figure}


Une Business Unit (BU) : Secteur public 
Le Secteur Public est le marché majoritaire chez Sopra Steria Group puisqu’il représente 25\% de son chiffre d’affaire.  La Business Unit du Secteur public est réparti sur 4 agences :  

\begin{itemize}
    \item Santé, Social, Emploi : Agence 151 (Sécurité sociale, Pole emploi, CNAMTS…), 
    \item echerche Enseignement : Agence 152 (ministère Éducation nationale, …), 
    \item dministration Centrale : Agence 156 (Mairie de Paris, DGFIP, ONP …), 
    \item onseil : Agence 155 (Clients transverses à toute la BU). Pour ma part, je travaille au sein de l’agence 151, pour le compte de la CNAM.
\end{itemize}


Au sein d'un même secteur d'activité, l'entreprise est divisée en agence qui fonctionnent comme des entreprises autonomes. Elles ont chacun un directeur d'agence, celui-ci dispose d'un grand pouvoir décisionnel au sein de son agence. Son objectif est que l'agence soit performante et fasse des bénéfices.
Une agence prend en charge de nombreux projets, dans notre cas, l'agence 151 gère les missions relevant du domaine de la Santé, du Social et de l’Emploi.
Ci-dessous la division au sein du secteur public.

\begin{figure}[!h]
\centering
\includegraphics[width=1\textwidth]{images/divisionSecteurPublic.png}
\caption{Sopra Steria : Les agences du Secteur Public}
\end{figure}

\section{L'agence 151}
L'agence 151 est répartie dans plusieurs villes, soit dans les locaux de Sopra Steria, soit directement chez le client.
Les projets sont pris en charge par des équipes, et une équipe peut travailler sur plusieurs projets simultanément, de même plusieurs équipes peuvent travailler sur un même projet. Les équipes sont généralement d’une dizaine de membres, parmi lesquels on retrouve les rôles de :
- Chef de Projet (ou Project Manager),
- Analyste d’affaire (ou Business Analyst),
- Architecte,
- Expert produit (ou Product Expert),
- Solution Builder,
- Commercial.
Mon équipe est installée avec d'autres équipes Sopra Steria dont le client est la Cnam dans les locaux de Sopra Steria dans la tour Cytiscope à Montreuil. 

\section{Le client : la Cnam}

COPIER COLLER A REFORMULER
Chez Sopra Steria Group, plus de quatre cent collaborateurs travaillent pour le compte de la CNAM qui génère plus de 40 Millions d’euros de chiffre d’affaire. Ce chiffre est le plus important de toute la BU, ce qui fait de la CNAM un client d’importance maximale pour la société. Pour en dire un peu plus que la Caisse Nationale d’Assurance Maladie (CNAM), elle gère les branches maladie du régime général de la sécurité sociale, et représente :

\begin{itemize}
    \item 57 millions de bénéficiaires affiliés au régime général, 
    \item 4 assurés sur 5, 
    \item 75\% des dépenses de santé. 
\end{itemize}

Sopra Steria Group cherche à aider toutes ces entités en même temps dans leurs tâches quotidiennes. Les équipes de développement web (CNAMTS METIER) et de développement BI (CNAMTS BI) travaillent pour atteindre ces objectifs. Je suis moi même rattaché à la CNAM Métier. Ci-dessous la liste des projets/équipes de la CNAM Métier :

\begin{itemize}
    \item ARPEGE
    \item BIC : Briques I C
    \item CS Nantes
    \item DMP
    \item DPO
    \item DPRA
    \item INDIGO
    \item PPIL : Portail PILotage.
    \item OVERSI
\end{itemize}

\section{Le projet PPIL}
\subsubsection{Un Portail de PILotage}
PPIL permet d'assurer le suivi des projets de la CNAM.
Celui-ci répond à plusieurs besoins :
\begin{itemize}
    \item Centralisation : au sein d’un même espace des données de planification et de pilotage venant de Microsoft Project
    \item Suivi & Contrôle : vues consolidées facilitant le suivi et la vérification des données de pilotage (actualisation des données, cohérence, …)
    \item Évolutivité & Maintenance : socle permettant de mettre en place de nouvelles fonctionnalités et de les déployer pour tous les utilisateurs
    \item Communication : faciliter la diffusion de l’information
\end{itemize}
L'application PPIL a fait l'objet d'une refonte en 2017, désignée sous le nom PPIL V2. 

Les objectifs de cette version sont :
Prioritairement :
\begin{itemize}
    \item Les nouveautés concernant la gestion des projets
    \item Le fonctionnement de l’application PPIL V2 sur socle SharePoint 2013,
    \item L’iso-fonctionnalité par rapport à l’application PPIL V1, à l’exception de quelques fonctionnalités modifiées ou supprimées.
Secondairement :
    \item La simplification du portail,
    \item Des accès aux principales fonctionnalités dès la connexion en fonction des profils utilisateurs,
    \item Un usage simplifié et plus clair des fonctionnalités,
    \item La gestion des droits,
    \item La gestion des référentiels simples,
    \item La gestion des demandes et des lots.
\end{itemize}

Le Portail Pilotage est composé de différents Espaces dont l’accès est conditionné par le profil de l’utilisateur.
Les droits en lecture et/ou écriture selon le profil de l’utilisateur sont visibles dans le document de gouvernance (Gouvernance ci dessous).

Les utilisateurs accèdent à « Mon espace » et selon leurs profils, les informations restituées à l’écran varient. Les sous-espaces définis pour chaque profil suivent la description ci-après, dans l’ordre défini 

\subsubsection{Utilisateurs de PPIL :}
Opérationnel, comprenant les : 
\begin{itemize}
    \item Responsable DSI
    \item MOA
    \item Managers (Responsable de Direction ou Responsable de Département) 
    \item Chef de Projet
\end{itemize}
Stratégiques, comprenant les :
\begin{itemize}
    \item PMO
    \item Responsable de domaines
\end{itemize}

\subsubsection{Les acteurs et leurs niveaux d'autorisation}
\begin{figure}[!h]
\centering
\includegraphics[width=0.8\textwidth]{images/ppil acteurs.png}
\caption{PPIL Les acteurs et leurs autorisations}
\end{figure}

\subsubsection{Lien LOT - Projet - PrtPalier}

(fig. \ref{fig:ppil lien lot projet prtpalier}).

\begin{figure}[!h]
\centering
\includegraphics[width=1\textwidth]{images/ppil lien lot projet prtpalier.png}
\caption{PPIL lien lot projet PrtPalier}
\end{figure}

\subsubsection{Principe du Reporting - Suivi de projet}

AU sein de la CNAM, tous les acteurs d’un lot renseignent et mettent à jour le planning MSP de leur projet, en renseignant des données de type :
\begin{itemize}
    \item Avancement charges, 
    \item Dates des jalons, 
    \item consommé des ressources.
\end{itemize}

Les informations MSP sont remontées dans PPIL de plusieurs manières :
\begin{itemize}
    \item Batch 2 fois par jour
    \item Données prévisionnelles
    \item Actualiser les données MSP (CP) 
\end{itemize}

Tous les acteurs doivent renseigner les risques et problèmes rencontrés sur leur projet ainsi que l'état de leur projet (facultatif).

\subsubsection{Principe du Reporting - CP}
Le CP du projet référent doit soumettre le reporting du lot toutes les 2 semaines (météo, tendance, situation).

Une fois soumis le reporting du lot est visible par tous les utilisateurs du PPIL.

Le Manager doit saisir la note de conjoncture du lot tous les 2 mois : 
Cette action permet d'expliquer la situation opérationnelle du lot de manière moins technique, cette note de conjoncture est plus destiné aux supérieurs hiérarchiques ( Resp DSI).

\subsubsection{Chef de Projet}
Dashboard du CP :

\begin{itemize}
    \item Bulletin de santé
    \item Dérive des jalons
    \item Plan de charges équipe
    \item Accès à la liste de mes projets en cours
    \item Note de conjoncture à mettre à jour
    \item Alertes
    \item Rapports
\end{itemize}

Accès au portail PPIL via une délégation
Un délégant, a la possibilité de donner l’accès à ses espaces à un autre utilisateur délégataire.
Le délégataire choisit dans la liste des personnes dans le bandeau, sous le profil, le profil du délégant. Le délégataire possède tous les droits sur l’espace de son délégant.

\subsubsection{Visualiser des indicateurs}

PPIL permet aux différents profils d'utilisateurs de visualiser les indicateurs opérationnels sur son périmètre.

Par exemple, pour les profils CP, Manager, Responsable DSI et MOA, on peut visualiser :

\begin{itemize}
    \item l’indicateur Bulletin de santé
Les projets dont la tendance est en dégradation et/ou la météo est orageuse sont mis en évidence par cet indicateur.
Visualiser l’indicateur Dérive des jalons
Les projets dont le prochain jalon et/ou la date de mise en production est en dérive sont mis en évidence par cet indicateur.
    \begin{itemize}
        \item Un jalon est en dérive lorsqu’il existe une différence de plus de 7 jours entre les dates du dernier reporting et de celui fait il y a un mois.
    \end{itemize}
    \item 
\end{itemize}

Visualiser l’indicateur Avancement par phase : Chef de projet
    Les jalons de tous les projets aux états « En cours » du périmètre sont représentés dans cet indicateur.

Visualiser l’indicateur Plan de charge équipe : Manager
En cliquant sur le graphique, on accède au rapport du Capacity Planning

Visualiser l’indicateur Planning des MEP : Responsable DSI et MOA
Les lots dont la date de mise en production est comprise entre le mois passé et les six prochains mois sont placés sur une échelle de temps.

Accès au reporting
Dans les indicateurs dans lesquels sont affichés les noms des projets, en cliquant sur le nom d’un projet :
	En tant que Manager, Responsable DSI et MOA, on accède à la restitution du reporting
	En tant que CP, on accède à la saisie du reporting

\subsubsection{Exécuter les actions rapides liées à mes projets}

Atteindre reporting et Actualiser les données MSP : Chef de projet

Visualiser le nombre de notes de conjoncture à mettre à jour : Manager


\subsubsection{Accéder à mes « informations opérationnelles » : CP, Manager, Responsable DSI et MOA}
Gérer les actions de mes projets : Chef de projet
Les actions remontées depuis une semaine ou moins sont taguées comme « Nouveau » et seules les actions dont la date d’échéance est soit dépassée, soit dont l’échéance est à venir sont affichées.
Consulter mes rapports opérationnels
En tant que CP, Manager, Responsable DSI et MOA, il est possible d’accéder au bloc « Rapports » dans « Mon espace » au niveau du sous-espace, situé dans la partie gauche de la page en dessous du bloc Alertes.
Visualiser la liste des projets / lots de mon périmètre


%%% Local Variables: 
%%% mode: latex
%%% TeX-master: "isae-report-template"
%%% End: 
\chapter{Ma mission au sein de PPIL}
\label{sec:unchapitre}
\section{[Objectifs de mission] Intégrer l'équipe PPIL et participer au développement du projet}
Ma mission au cours de ces deux premiers mois de stage a été d'intégrer l'équipe PPIL et de participer au développement du projet. Mes objectifs de mission s'articule autour de trois points :
\subsection{Développement :} 
\begin{itemize}
    \item Acquérir les compétences techniques nécessaires au projet.
    \item Corriger les anomalies affectées par le référent technique sur le périmètre PPIL 
    \item Aucun retour bloquant en qualification interne et externe 
    \item Garantir la non régression
\end{itemize}
\subsection{Qualification : } 
\begin{itemize}
    \item Participer à l'exécution des tests internes : rigueur dans l'exécution des tests.
    \item Remonter les anomalies détectées. Qualifier correctement une anomalie (intitulé, criticité, priorité, type, module, ...) de manière à rendre le plus compréhensible possible l'anomalie, pour le développeur.
\end{itemize}
\subsection{Reporting et Autonomie :} 
\begin{itemize}
    \item Estimation et Respect des délais et des charges
    \item --> Estimer ses charges, suivre son RAE, et le cas échéant expliquer les dérives 
    \item Reporting :
    \item --> Assurer le reporting auprès de son tuteur et remonter les difficultés rencontrées
    \item Capitaliser : 
    \item --> Capitaliser ses travaux sur le wiki et le groupe réseau
\end{itemize}
\section{Fonctionnement de l'équipe}
\subsection{Les différents rôles}
J'ai intégré l'équipe de développement du Portail de PILotage (PPIL) utilisé par les chefs de projet (et autres profils) de la Cnam. L'équipe est composée de 12 collaborateurs. Elle est notamment constituée de :
\begin{itemize}
    \item un CP (chef de projet) 
    \item un RT (référent technique)
    \item un RF (référent fonctionnel)
    \item des BA (business analyst)
    \item des SB (solution builder)
\end{itemize}
Le RT (référent technique) sont les référents fonctionnels, ils gèrent la répartition et l'avancement des tâches de chacun (RAE), il est en contact permanent avec les dév et le CP. Le RF est lui en contact direct avec le client, les BA et le CP.
\subsection{Les processus mis en place}
\subsubsection{processus de livraison}
Les SB (développeurs) développent sur l'environnement de développement.
Les BA disposent de deux environnements de qualification, ils peuvent donc réaliser des tests sur deux versions de PPIL.
A la fin d'un sprint, lorsque les développements évolutions du lot sont terminées, le lot est livré lors d'une "livraison interne" dans l'environnement "qual" de qualification ou les BA peuvent tester la nouvelle version de PPIL.
livraison client = le client teste de son coté avec son équipe;
MEP A la fin d'un lot, mise en production signifie la livraison
sprint de 3 semaines
\subsubsection{processus au sein de l'équipe}
daily meeting tout les jours, 
V1 PPIL, réunions hebdomadaires qui rassemble toute l'équipe PPIL.
\begin{itemize}
    \item point sur l'Agence 151 / plateau (le plateau contient une partie de l'agence), des autres projets de l'agence
    \item point sur PPIL
    \item rétrospective de la semaine, santé, tendance, satisfaction du client
    \item évolutions du projet, attribution des taches
\end{itemize}
%%% Local Variables: 
%%% mode: latex
%%% TeX-master: "isae-report-template"
%%% End: 
\chapter{Démarche adoptée et réalisations au sein de PPIL}
\label{chap:premierchapitre}

% --> A ajouter : 
% présentation de PPIL
% indicateurs intéressants. dans montée en compétence : Un projet de gestion de projet
% retour PMO
% mieux rédiger TNR
% mieux rédiger ANO
% vu toute la production d'un projet
%CNAM Métier
%HP ALM

\section{Montée en compétence sur le fonctionnel du projet}

Pendant ma première semaine de stage, j'ai effectué une montée en compétence sur le fonctionnel du projet. Il est important de souligner que j'ai continué à en apprendre toujours d'avantage sur le fonctionnel du projet tout le long de mon stage que ce soit en qualification ou en développement.

Dans cette section, je vais décrire ma démarche pour m'imprégner du projet. Il est évident que cette montée en compétence a été primordiale pour la suite de mon stage.

Pour cela, j'ai utilisé plusieurs méthodes :

Tout d'abord, j'ai étudié les spécifications fonctionnelles et technique du Projet, je me suis aussi procuré le manuel utilisateur de l'application auprès de l'équipe.

Au cours de cette montée en compétence, j'ai posé des questions aux différents analystes fonctionnels du projet, j'ai donc pu bénéficier de les explications des sachant sur le projet : le client, le contexte du projet ainsi que l'histoire de PPIL. 

Quelques informations importantes que j'ai pu recueillir lors de mon arrivée dans l'équipe :
\begin{itemize}
    \item Début de PPIL en 2010, 
    \item Refonte du projet en 2017, 
    \item Exigences du client,
    \item Des sprints qui durent 3 semaines, 
    \item Les besoins et les outils du client (les différents acteurs de la Cnam utilisent MSP),
    \item Comprendre les différentes fonctionnalités de PPIL,
    \item PPIL et les autres projets de la CNAM : on retrouve dans PPIL tous les autres projets de la Cnam,
    \item PPIL est intégré dans SharePoint car les agents de la CNAM manipulent des documents dans leur Sharepoint et le fait d'avoir PPIL dans une même structure leur permet de rassembler plusieurs outils au même endroit.
    \item Fonctionnement du Reporting de PPIL, 
    \item Différents concepts : indicateurs, jalons, diagrammes, lots, projet de référence, projets contributeurs.
\end{itemize}

En plus de ces explications et cette étude des documents, j'ai pu manipuler l'application PPIL, la prendre en main afin de me mettre à la place des utilisateurs finaux et de comprendre comment et pourquoi ils utilisent cet outil qu'est PPIL.

Il a été important de me mettre à la palce des différents profils d'utilisateurs de PPIL :
\begin{itemize}
    \item Chef de projet
    \item Manager
    \item Responsable DSI...
\end{itemize}

A la fin de la semaine, j'ai réalisé une présentation du projet et de ses fonctionnalités aux membres de mon équipe. La présentation a duré 10 minutes, cette présentation a eu pour objectif de présenter tout ce que j'ai appris sur le projet durant la semaine. A la suite de cette présentation, nous avons échangé avec les BA, RT et le chef de projet afin de préciser certains points important dans l'objectif d'en l'objectif que j'en sache un maximum sur le projet.

\subsection{Des indicateurs intéressants en terme de gestion de projet}

Dans cette partie je vais expliquer pourquoi PPIL est très intéressant en terme de gestion de projets. Il permet de voir comment une organisation (la Cnam) gère une multitude de projets.

J'ai découvert dans PPIL le mécanisme de reporting de projets et de lots ainsi que des indicateurs qui permettent de :
\begin{itemize}
    \item Planifier les projets dans le temps (notamment grace aux concept de jalon),
    \item Maîtriser et piloter les risques,
    \item Gérer un grand nombre de projets,
    \item Suivre des enjeux opérationnels de projets ou de lots,
    \item S'adapte en fonction des différents acteurs intervenant dans la gestion de projets.
\end{itemize}

\subsubsection{Visualiser les indicateurs en fonction du profil}

PPIL permet aux différents profils d'utilisateurs de visualiser les informations opérationnelles de leurs projets.

Pour les profils Chef de projet, Manager, Responsable DSI et MOA, on peut visualiser :
\begin{itemize}
    \item Visualiser l’indicateur \textbf{Bulletin de santé}
    Les projets dont la tendance est en dégradation et/ou la météo est orageuse sont mis en évidence par cet indicateur.
    \item Visualiser l’indicateur \textbf{Dérive des jalons}
    \begin{itemize}
        \item Les projets dont le prochain jalon et/ou la date de mise en production est en dérive sont mis en évidence par cet indicateur.
        \item Un jalon est en dérive lorsqu’il existe une différence de plus de 7 jours entre les dates du dernier reporting et de celui fait il y a un mois.
    \end{itemize}
    \item Visualiser l’indicateur \textbf{Avancement par phase} (Chef de projet)
    Les jalons de tous les projets aux états « En cours » du périmètre sont représentés dans cet indicateur.
    \item Visualiser l’indicateur Plan de charge équipe (Manager)
    En cliquant sur le graphique, on accède au rapport du Capacity Planning
    \item Visualiser l’indicateur \textbf{Planning des MEP} (mise en production) : Responsable DSI et MOA
    Les lots dont la date de mise en production est comprise entre le mois passé et les six prochains mois sont placés sur une échelle de temps.
\end{itemize}

\begin{figure}[h]
\centering
\includegraphics[width=1\textwidth]{images/ppil-bulletion-de-sante.PNG}
\caption{PPIL : Bulletin de santé (Profil Chef de Projet)}
\end{figure}

\subsubsection{Accéder aux reporting / restitution d'un lot / projet}
Pour rappelle, un lot contient plusieurs projets. Dans les indicateurs dans lesquels sont affichés les noms des projets, en cliquant sur le nom d’un projet :
\begin{itemize}
    \item En tant que Manager, Responsable DSI et MOA, on accède à la restitution du reporting
    \item En tant que CP, on accède à la saisie du reporting
\end{itemize}
En accédant au reporting d'un projet, on accède à différents indicateurs intéressants pour un lot/projet : 

\begin{figure}[!h]
\centering
\includegraphics[width=1\textwidth]{images/PPIL-avancement.png}
\caption{PPIL : Graphique d'avancement en charges}
\end{figure}


\begin{figure}[!h]
\centering
\includegraphics[width=2.5\textwidth]{images/PPIL-Gantt.png}
\caption{PPIL : Gantt des projets contributeurs}
\end{figure}

\begin{figure}[H]
\centering
\includegraphics[width=0.8\textwidth]{images/temps temps.png}
\caption{PPIL : Capacity Planning}
\end{figure}

\subsubsection{Exécuter les actions rapides liées à mes projets}
Les actions rapides qui peuvent être effectués en tant que chef de projet sont \textbf{atteindre reporting} et \textbf{Actualiser les données MSP}, et en temps que Manager, on peut visualiser le nombre de notes de conjoncture à mettre à jour.

\subsubsection{Accéder à mes « informations opérationnelles » (CP, Manager, Responsable DSI et MOA)}
Gérer les actions de mes projets
Consulter les actions
Consulter les Facteurs de risque / problèmes
Consulter mes rapports opérationnels
Accéder aux « Rapports »


\section{La qualification}

Une partie de mon stage est destinée à la qualification, j'ai effectué différents tests :
\begin{itemize}
    \item Tests fonctionnels
    \item Tests de Non Régression
    \item Plan de tests
\end{itemize}

\subsection{L'importance des tests}

Durant mon stage, j'ai réalisé des tests afin de vérifier le bon fonctionnement des nouvelles fonctionnalités et corrections du projet ayant pour objectif de détecter d'éventuels anomalies ou régressions. Ces tests m'ont été bénéfique pour comprendre PPIL en profondeur. En effet, pour chaque tests, il faut comprendre et manipuler la base de données du projet, consulter les spécifications techniques ou fonctionnelles (SFG, STD) et poser des questions aux différents membres de l'équipe.

Au cours de ces tests, j'ai rencontré plusieurs difficultés : 

\begin{itemize}
    \item Certaines fonctionnalités à tester ne sont pas évidente à comprendre ou à reproduire.
    \item Les principes de l'outil : par exemple la logique de reporting en fonction des différents indicateurs.
    \item Manipuler une base de donnée complexe. (préparer des jeux de données, changer d'utilisateur, vérifier des informations en base)
\end{itemize}

A chaque fois, ces tests ont été effectués avant une livraison interne ou une livraison client.

Lors des tests il est important de prendre du recul et de tester d'autres fonctionnalités qui pourraient être impactées par la correction qu'on est en train de tester afin de détecter d'éventuels régressions.

Tous ces tests ont été réalisé grâce à l'outil HPALM.
        
Valider un test c'est prendre la responsabilité de dire qu'on peut livrer l'application. L'étape du test est primordiale, sans celle-ci un bon nombre d'erreurs ne seraient pas détectés.

\subsection{Les plans de tests}

J'ai eu l'occasion de rédiger des plans de tests. Cette étape demande une grande rigueur car il est important de couvrir tout le périmètre de la fonctionnalité à tester pour découvrir d'éventuels régressions. Ci-dessous un exemple de plan de test :

\begin{figure}[!h]
\centering
\includegraphics[width=0.8\textwidth]{images/HPLMplantest.png}
\caption{HP LM : Plan de test}
\end{figure}

\subsection{Un exemple de test réalisé : Les tests de Non Régression (TNR)}

[semaine 11/06]
Avant la livraison de la release 20.06.00. Il a fallut effectuer des tests de non régression.

Je vais décrire ici le test "06-TNR-ReportingRestitutionDuLot". Lors de ce test, j'ai détecté trois Defects. Ce qui a permis d'identifier et corriger une anomalie mineure, une anomalie majeure, et une régression majeure. Au cours de ce test de 40 steps, j'ai effectué un grand nombre de requêtes SQL, j'ai dû comprendre la logique de calcul ainsi que la logique d'affichage des indicateurs d'avancement d'un projet en fonction des jalons de celui-ci. Il a fallu que j'analyse la synchronisation des données projet entre Microsoft Project et PPIL.

\subsubsection{Difficultés rencontrées lors de ce test}
%capt ecran

description du step de ce test : 
vérifier que les indicateurs de suivi d'avancement des lots sont corrects :
Dans PPIL, dans la partie restitution de l'avancement d'un projet de référence. Dans cette partie il a fallu comprendre comment sont calculés les indicateurs de restitution pour l'avancement d'un lot. Pour cela j'ai consulté les SFD pour voir les règles de calcul. Je suis allé dans le code chercher les procédures stockées, après cela, j'ai fait mes calculs à l'aide de Excel et de SQL, n'ayant toujours pas le même résultat, j'ai demandé des explications à un BA notamment sur la synchronisation des données de MIcrosoft Project dans PPIL(outil utilisé par les CP de la Cnam pour mettre à jour les jalons de leurs projets). Et j'ai analysé une procédure stockée. Grâce à ces explications et à ces analyses j'ai réussi à ajuster ma requête. 


%Expliquer ce problème
De restitution qui importe les projets contributeurs (qui font parti du même PRT Palier).

Pour la partie Avancement en charge
\begin{figure}[h]
\centering
\includegraphics[width=0.8\textwidth]{HPALM-test.PNG}\\[1cm]
\caption{HP ALM : Déroulement d'un test}
\end{figure}

\section{Environnement technique du projet}

Pendant la deuxième semaine, j'ai également mis en place mon environnement de développement avec Achref (mon SB référent). Le projet PPIL se situe dans un contexte technique Microsoft. 

\begin{itemize}
    \item Sharepoint, 
    \item SQLServer, 
    \item .net
    \begin{itemize}
        \item C\#, 
        \item Telerik, 
        \item TypeScrypt, 
        \item Entity Framework, 
        \item SSRS, 
        \item HTML, 
        \item CSS etc).
    \end{itemize}
\end{itemize}

Outils utilisés :
\begin{itemize}
    \item Visual Studio
    \item HPLM
    \item Git
    \item Microsoft SQL Server Management
    \item Microsoft Team Foundation Server
\end{itemize}

\section{Développer au sein de PPIL}

J'ai commencé à développer au mois de Mai. Les développements au sein de PPIL se font en fonction de l'évolution du sprint en cours. Je suis arrivé dans un contexte de correction d'anomalie. C'est naturellement que des (QC) corrections d'anomalies m'ont été attribuées.

Ma première tâche a été de normaliser la charte graphique en même temps que l'interface du projet dans le but d'harmoniser les deux. Cette tâche fait suite à un retour du client (FT). Cela m'a permis de découvrir l'environnement technique. 

Ensuite, j'ai principalement corrigé des anomalies et développé quelques évolutions.

\subsection{Correction d'anomalies}

Lorsque les BA détectent une anomalie, ils l'identifient et la répertorie dans l'outil HPALM. Il est important de préciser que certaines anomalies sont détectées par le client. Elles sont classées par priorité et importance.
Les FT (anomalie retour client) sont souvent prioritaires par rapport aux anomalies détectés par l'équipe.
Un lot correspond à une version de l'application, chaque defect est rattaché à un lot.

\begin{figure}[h]
\centering
\includegraphics[width=1\textwidth]{images/HPLMliste.png}
\caption{HP LM : Defects corrigés de la release 20.06.00}
\end{figure}

\subsection{Démarche Adoptée}

Avant de commencer le développement, j'ai dû lire les consignes de code, règles à respecter soigneusement car mes développements seront livrés directement.

J'ai été chargé de corriger les anomalies distribuées par les RT.
Le référent technique a attribué différentes QC (Anomalies) aux différents SB. 
Les anomalies que j'ai corrigées étaienent toutes différentes et incluaient différentes technologies à chaque fois.

Quelques exemples d'anomalies que j'ai corrigés : 
\begin{itemize}
    \item Suppression d'un élément qui ne se supprime pas en base
    \item Erreur dans le chargement d'une page
    \item Bouton "annuler" non présent
    \item Ordre ou classement pas correct
    \item Direction de page incorrect lors de la consultation...
\end{itemize}

Pour mener à bien ces différentes corrections, il a été important de prendre du recul avant chaque correction. De bien identifier le périmètre du defect afin d'éviter les régressions.

Tout d'abord, j'ai du comprendre et analyser l'architecture du programme afin d'identifier plus facilement la provenance des defects. 

Chacun des points suivant ont été très important dans ma démarche de correction d'anomalie :
\begin{itemize}
    \item étudier l'architecture de la BDD a été primordiale pour avoir une visibilité sur les relations entre les différentes entités 
    \item me former sur les technos
    \item comprendre l'anomalie aussi bien fonctionnellement que techniquement
    \item analyser l'ampleur et périmètre de l'anomalie
    \begin{itemize}
        \item pour quel type d'utilisateur 
        \item dans quel mode (consultation, restitution, 
        \item dans quel rubrique ( mes projets, ptf, ...
    \end{itemize}
    \item identifier la provenance du problème

\end{itemize}

Comprendre la logique de développement :
\begin{itemize}
    \item comprendre pourquoi le projet a été codé de cette manière
    \begin{itemize}
        \item aller voir les développeurs, 
        \item aller voir le référent technique,
        \item consulter spécifications techniques ou fonctionnelles
    \end{itemize}
    \item identifier les solutions possibles
    \item choisir la meilleure façon de faire à :
    \begin{itemize}
        \item modifier au minimum le code,
        \item trouver la solution la plus optimale,
        \item éviter faire de régression 
    \end{itemize}
 \end{itemize}              
            
\subsubsection{Respect des délais (RAE)}
Pour chacune de mes tâches j'ai dû estimer le temps prévu à la réalisation de celle-ci. Respecter les délais est très important sur plusieurs points :
\begin{itemize}
    \item Pour le projet, cela sert de savoir quand on pourra proposer une version du projet au client.
    \item Pour l'équipe afin de savoir quelle tâche est assignée à quel développeur. 
    \item Pour moi, dans le choix de la démarche de résolution de la tâche.
\end{itemize}

Processus et outil utilisés :
\begin{itemize}
    \item Après avoir développé une évolution les étapes à suivre sont :
    \item "Pusher" ma branche avec git sur le dépot distant,
    \item Créer une "pull request" pour alerter le référent technique,
    \item Commenter la correction ou l'évolution sur HPLM
\end{itemize}

\begin{figure}[!h]
\centering
\includegraphics[width=1\textwidth]{images/PullRequest.png}
\caption{Visual Studio Team Foundation : Les Pull Request}
\end{figure}

Ci-dssous le workflow des anomalies dans HP ALM :
\begin{figure}[!h]
\centering
\includegraphics[width=0.6\textwidth]{images/DefectHPALM.png}
\caption{HP ALM Defect Workflow}
\end{figure}

\subsection{Quelques exemples d'anomalies corrigées}

Toutes les anomalies ont été intéressantes à corriger, et elles m'ont fait progresser et travailler sur des technologies diverses. J'ai réussi à acquérir les compétences techniques et nécessaires au projet. L'enjeu de la correction de ces anomalies c'est de n'avoir aucun retour bloquant en qualification interne et externe ainsi que de garantir la non régression.

\subsubsection{Anomalie [Poretefeuille][liste déroulante]}

Visuel de l'anomalie dans HPALM :
\begin{figure}[!h]
\centering
\includegraphics[width=1\textwidth]{images/QC2120.PNG}
\caption{HPALM : Defect 2112}
\end{figure}

On peut voir que l'anomalie est de priorité urgente et de gravité majeure, et qu'elle a été détectée par le client (priorité supplémentaire).

\textbf{Description de la correction à effectuer :} 

[Profil CP ou Manager][portefeuille]
Dans la liste déroulante qui apparaît lorsqu'on veut ajouter des projets ou des lots dans les différents portefeuilles de : 

\begin{itemize}
    \item lots statiques,
    \item lots dynamiques,
    \item projets statiques,
    \item projets dynamiques.
\end{itemize}

Sélectionner par défaut un portefeuille s'il n'y en a qu'un seul. Ne rien sélectionner par défaut sinon. 
[Profil Responsable DSI][portefeuille], Sélectionner par défaut "Mes projet favori" quelque soit le nombre de portefeuille existant.

Avant de commencer la correction du code de cette QC, j'ai mis au courant le référent technique que la description du defect était sujette à interprétation.Le RT a remonté l'information au client, qui a fait un retour en précisant le defect.

Lors de la correction en elle-même, j'ai du identifier ou faire les modification de code. Puis au cours de celle-ci j'ai pu monter en compétence sur du C\#, du Type-script ainsi que Kendo. J'ai du prendre en compte les différentes conditions pour réaliser différentes actions. 

\begin{figure}[!h]
\centering
\includegraphics[width=1\textwidth]{images/correction1.PNG}
\caption{HPALM : Defect 2112}
\end{figure}

\subsubsection{Anomalie [Supression des lots non effective dans plusieurs cas]}

\textbf{Description de la correction à effectuer :} 
La suppressions des lots non effective selon certains critères :
\begin{itemize}
    \item Un lot ne se supprime pas quand il est dans un portefeuille (le lot se supprime visuellement mais pas en base, quand on refresh la page il réapparaît)
    \item Un lot ne se supprime pas quand il a des demandes rattaché
    \item Un lot ne se supprime pas quand il n'a pas de projet de référence
\end{itemize}

Cette anomalie m'a pris plusieurs jours de correction, les points importants à souligner sont : 
\begin{itemize}
    \item Il a fallut mettre en place plusieurs jeux de données pour visualiser l'anomalie.
    \item Pendant la correction j'ai détecté une autre anomalie : un lot sans projet de référence ne s'affiche pas dans un portefeuille. (Anomalie que j'ai corrigé par la suite.)
    \item J'ai proposé une solution efficace et correcte.
    \item Je suis allé voir Nicolas (le RT du projet), afin d'avoir son retour sur ma correction.
    \item Il m'a apporté son recul et son expérience en me proposant de refaire une partie de cette correction en modifiant une procédure stockée en SQL plutôt que de modifier une partie du code. Cette solution étant préférable pour la maintenabilité du code et de l'optimisation**.
\end{itemize}

Pour rappel, ce sont les PMO qui ont les droits pour supprimer des lots, un lot est constitué d'un ensemble de projets. Ci-dessous l'interface des PMO.

\begin{figure}[!h]
\centering
\includegraphics[width=1\textwidth]{images/ano2.png}
\caption{HPALM : Defect 2112}
\end{figure}


\subsubsection{Troisième anomalie }

\textbf{Description de l'anomalie :} En restitution d'un projet, onglet synthèse, quand on remonte sur une semaine précédente, les dates ne sont pas les bonnes. 

\begin{figure}[!h]
\centering
\includegraphics[width=1\textwidth]{images/ppil-indicateur-synthese.PNG}
\caption{Indicateur Synthèse}
\end{figure}

Les difficultés rencontrées lors de cette correction ont été :
\begin{itemize}
    \item Comprendre le fonctionnel, c'est à dire la logique des jalons.
    \item Travailler et modifier des requêtes complexes.
    \item Ne pas faire de régression.
\end{itemize}


\subsubsection{Des anomalies très complexes ** photo ?}

des requêtes très complexes qui vont chercher des données sur le CUBE, aucun membre de l'équipe ne maîtrise la techno, je suis chargé pendant une semaine d'analyser les ano et de trouver une correction. Un des développeur m'a rejoint pour chercher des solutions avec moi.


%%% Local Variables: 
%%% mode: latex
%%% TeX-master: "isae-report-template"
%%% End: 
\chapter{Démarche adoptée et réalisations au sein de PPIL}
\label{chap:premierchapitre}

% --> A ajouter : 
% présentation de PPIL
% indicateurs intéressants. dans montée en compétence : Un projet de gestion de projet
% retour PMO
% mieux rédiger TNR
% mieux rédiger ANO
% vu toute la production d'un projet
%CNAM Métier
%HP ALM

\section{Montée en compétence sur le fonctionnel du projet}

Pendant ma première semaine de stage, j'ai effectué une montée en compétence sur le fonctionnel du projet. Il est important de souligner que j'ai continué à en apprendre toujours d'avantage sur le fonctionnel du projet tout le long de mon stage, que ce soit en qualification ou en développement.

Il est évident que cette montée en compétence a été primordiale pour la suite de mon stage. Dans cette section, je vais décrire ma démarche pour m'imprégner du projet. Pour cela, j'ai utilisé plusieurs méthodes :

Tout d'abord, j'ai étudié les spécifications fonctionnelles et techniques du projet (SFG, SFD et STD), je me suis aussi procuré le manuel utilisateur de l'application auprès de l'équipe.

Au cours de cette montée en compétence, j'ai posé des questions aux différents analystes fonctionnels du projet. J'ai donc pu bénéficier de leurs explications.

Quelques informations importantes que j'ai pu recueillir lors de mon arrivée dans l'équipe :
\begin{itemize}
    \item Début de PPIL en 2010 ;
    \item Refonte du projet en 2017 ;
    \item Exigences du client ;
    \item Des sprints qui durent 3 semaines ;
    \item Les besoins et les outils du client (les différents acteurs de la CNAM utilisent MSP) ;
    \item Comprendre les différentes fonctionnalités de PPIL ;
    \item PPIL et les autres projets de la CNAM : on retrouve dans PPIL tous les autres projets de la CNAM ;
    \item PPIL est intégré dans SharePoint dans le but d'avoir une meilleure organisation interne ;
    \item Fonctionnement du Reporting de PPIL ;
    \item Différents concepts : indicateurs, jalons, diagrammes, lots, projet de référence, projets contributeurs ; 
\end{itemize}

En plus de ces explications et cette étude des documents, j'ai pu manipuler l'application PPIL, la prendre en main afin de me mettre à la place des utilisateurs finaux et de comprendre comment et pourquoi ils utilisent cet outil qu'est PPIL.

A la fin de la semaine, j'ai réalisé une présentation du projet et de ses fonctionnalités aux membres de mon équipe. La présentation a duré dix minutes. Cette présentation a eu pour objectif de présenter tout ce que j'ai appris sur le projet durant la semaine. A la suite de cette présentation, nous avons échangé avec les BA, RT et le chef de projet afin de préciser certains points importants dans l'objectif d'en savoir un maximum sur le projet.

\subsection{Des indicateurs intéressants en terme de gestion de projet}

Il est évident que PPIL est très intéressant en terme de gestion de projets. Il permet de voir comment une organisation (la CNAM) gère une multitude de projets. Et aussi de voir comment sont gérés les différentes phases d'un projet.

J'ai découvert dans PPIL le mécanisme de reporting de projets et de lots ainsi que des indicateurs qui permettent de :
\begin{itemize}
    \item Planifier les projets dans le temps (notamment grâce aux concepts de jalons) ;
    \item Maîtriser et piloter les risques ;
    \item Gérer un grand nombre de projets ;
    \item Suivre des enjeux opérationnels de projets ou de lots ;
    \item S'adapter en fonction des différents acteurs intervenants dans la gestion de projets.
\end{itemize}
\vspace{\baselineskip}
L'objectif dans cette partie est de présenter \textbf{quelques indicateurs} visibles dans PPIL. C'est un portail qui est peut être en phase de devenir \textbf{une référence en terme de gestion de projets}. A fortiori pour moi, étudiant en informatique. 

\subsubsection{Quelques indicateurs}

PPIL permet aux différents profils d'utilisateurs de visualiser les informations opérationnelles de leurs projets.

Pour les profils Chef de projet, Manager, Responsable DSI et MOA, on peut :
\begin{itemize}
    \item Visualiser l’indicateur \textbf{Bulletin de santé} : les projets dont la tendance est en dégradation et/ou la météo est orageuse sont mis en évidence par cet indicateur ;
    %\item Visualiser l’indicateur \textbf{Dérive des jalons}
    %\begin{itemize}
        %\item Les projets dont le prochain jalon et/ou la date de mise en production est en dérive sont mis en évidence par cet indicateur.
        %\item Un jalon est en dérive lorsqu’il existe une différence de plus de 7 jours entre les dates du dernier reporting et de celui fait il y a un mois.
    %\end{itemize}
    \item Visualiser l’indicateur \textbf{Avancement par phase} (Chef de projet) : les jalons de tous les projets aux états « En cours » du périmètre sont représentés dans cet indicateur ;
    \item Visualiser l’indicateur Plan de charge équipe (Manager) ;
    %En cliquant sur le graphique, on accède au rapport du Capacity Planning
    %\item Visualiser l’indicateur \textbf{Planning des MEP} (mise en production) : Responsable DSI et MOA
    %Les lots dont la date de mise en production est comprise entre le mois passé et les six prochains mois sont placés sur une échelle de temps.
\end{itemize}

\begin{figure}[h]
\centering
\includegraphics[width=1\textwidth]{images/ppil-bulletion-de-sante-censored.png}
\caption{PPIL : Bulletin de santé (Profil Chef de Projet)}
\end{figure}

\subsubsection{Accéder aux reporting / restitution d'un lot / projet}

Pour rappel, un lot contient plusieurs projets. En cliquant sur le nom d’un projet/lot, les indicateurs s'affichent soit en saisie, soit en restitution, en fonction des profils. 

En accédant au reporting d'un projet, on accède à différents indicateurs intéressants pour un lot/projet : 

Ci-dessous on peut retrouver l'avancement par phases des projets d'un lot ainsi que détail de toutes les phases de réalisation des projets. On peut voir l'évolution de chacun d'eux (en haut on peut voir le projet de référence).

\begin{figure}[h]
\centering
\includegraphics[width=1\textwidth]{images/PPIL-Gantt-censored.png}
\caption{PPIL : Diagramme de Gantt / Avancement par phase}
\end{figure}

Ci-dessous on peut visualiser l'avancement en charges qui nous donne des indicateurs de performance d'un lot (c'est à dire un ensemble de projet) : on peut noter que l'indicateur "avancement en charges" est calculé à partir du "consommé global" et du "RAF". Le Forecast est calculé à partir du : "Consommé global", du "RAF" et des Charges. J'ai du vérifier tous les indicateurs du graphique d'avancement en charges lors d'une des phases de qualification (pendant les tests de non régression).
\begin{figure}[h]
\centering
\includegraphics[width=1\textwidth]{images/PPIL-avancement.png}
\caption{PPIL : Graphique d'avancement en charges}
\end{figure}

L'indicateur ci-dessous précise l'avancement d'un projet en fonction des différents jalons. On peut voir les jalons dont la date a été franchie, reportée ou en retard. 

\begin{figure}[H]
\centering
\includegraphics[width=1\textwidth]{images/temps-temps.png}
\caption{PPIL : Diagramme temps-temps}
\end{figure}

\section{La qualification}

Une partie de mon stage est destinée à la qualification. J'ai effectué différents tests :
\begin{itemize}
    \item Tests fonctionnels ; 
    \item Tests de Non Régression ;
    \item Plan de tests ;
\end{itemize}

\subsection{L'importance des tests}

J'ai réalisé des tests afin de vérifier le bon fonctionnement des nouvelles fonctionnalités. Ainsi que des corrections du projet ayant pour objectif de détecter d'éventuels anomalies ou régressions. Ces tests m'ont étés bénéfiques pour comprendre PPIL en profondeur. En effet, pour chaque test, il faut comprendre et manipuler la base de données du projet, consulter les spécifications techniques ou fonctionnelles (SFG, STD) et poser des questions aux différents membres de l'équipe.

Au cours de ces tests, j'ai rencontré plusieurs difficultés : 

\begin{itemize}
    \item Certaines fonctionnalités à tester ne sont pas évidentes à comprendre ou à reproduire.
    \item Les principes de l'outil (par exemple la logique de reporting en fonction des différents indicateurs).
    \item Manipuler une base de données complexe (préparer des jeux de données, changer d'utilisateur, vérifier des informations en base).
\end{itemize}

A chaque fois, ces tests ont été effectués avant une livraison interne ou une livraison client.

Lors des tests, il est important de prendre du recul et de tester d'autres fonctionnalités. Le portail pourrait être impacté par la correction qu'on est en train de tester. Tous ces tests ont été réalisé grâce à l'outil HPALM.
        
Valider un test c'est prendre la responsabilité de dire qu'on peut livrer l'application. L'étape du test est primordiale. Sans celle-ci un bon nombre d'erreurs ne seraient pas détectées.

\subsection{Les plans de tests}

J'ai eu l'occasion de rédiger des plans de tests. Cette étape demande une grande rigueur car il est important de couvrir tout le périmètre de la fonctionnalité à tester pour découvrir d'éventuelles régressions. Ci-dessous, un exemple de plan de test :

\begin{figure}[!h]
\centering
\includegraphics[width=0.8\textwidth]{images/HPLMplantest.png}
\caption{HPALM : Un plan de test}
\end{figure}

\subsection{Un exemple de test réalisé : Les tests de Non Régression (TNR)}

Avant la livraison de la release 20.06.00. Il a fallu effectuer des tests de non régression.

Je vais décrire ici le test "06-TNR-ReportingRestitutionDuLot". Lors de ce test, j'ai détecté trois Defects. Ce qui a permis d'identifier et corriger une anomalie mineure, une anomalie majeure, et une régression majeure. Au cours de ce test de 40 steps, j'ai effectué un grand nombre de requêtes SQL. J'ai dû comprendre la logique de calcul ainsi que la logique d'affichage des indicateurs d'avancement d'un projet en fonction des jalons de celui-ci. Il a fallu que j'analyse la synchronisation des données projet entre Microsoft Project et PPIL.

\subsubsection{Difficultés rencontrées lors de ce test}

\textbf{Description d'une étape qui m'a posée des difficultés pendant ce test :}
C'est l'étape de vérification des indicateurs de suivi d'avancement du lot que j'ai choisi de détailler.
Le cas à vérifier se trouve dans PPIL, dans la partie restitution de l'avancement d'un projet de référence. Mon but étant de vérifier que les résultats restitués dans les indicateurs soient bien correct. Pour cela il a fallu que j'aille chercher les données en base et que j'effectue des calculs avec ces données.
Pour comprendre comment sont calculés les indicateurs de restitution pour l'avancement d'un lot, j'ai consulté les SFD afin de récupérer les règles de calcul. Je suis allé en base de données afin de récupérer les données (récupérer les projets à partir desquels les calculs sont effectués). J'ai du faire des requêtes assez complexes pour récupérer les bons projets. Après cela, j'ai importé les projets et leurs informations dans Excel pour y faire mes calculs. N'ayant toujours pas le bon résultat, mais ayant fait ces calculs avec rigueur, j'ai demandé des explications à un BA, notamment, sur un détail fonctionnel particulier. Ce détail portait sur la synchronisation des données de Microsoft Project dans PPIL (outil utilisé par les Chef de projet de la CNAM pour mettre à jour les jalons de leurs projets). J'ai finalement trouvé la solution. Grâce aux explications et à l'analyse d'une procédure stockée.

%Pour résumer, j'ai du faire appel à plusieurs
%Expliquer ce problème
%De restitution qui importe les projets contributeurs (qui font parti du même PRT Palier).
%Pour la partie Avancement en charge
\begin{figure}[h]
\centering
\includegraphics[width=0.8\textwidth]{HPALM-test.PNG}\\[1cm]
\caption{HPALM : Déroulement d'un test}
\end{figure}

\section{Environnement technique du projet}

Pendant la deuxième semaine, j'ai également mis en place mon environnement de développement avec Achref (mon SB référent). Le projet PPIL se situe dans un contexte technique Microsoft : 

\begin{itemize}
    \item Sharepoint, 
    \item SQLServer, 
    \item .Net
    \begin{itemize}
        \item C\#, 
        \item Telerik, 
        \item TypeScrypt, 
        \item Entity Framework, 
        \item SSRS, 
        \item HTML, 
        \item CSS.
    \end{itemize}
\end{itemize}

Outils utilisés :
\begin{itemize}
    \item Visual Studio,
    \item HPALM,
    \item Git,
    \item Microsoft SQL Server Management,
    \item Microsoft Team Foundation Server.
\end{itemize}

\section{Développer au sein de PPIL}

J'ai commencé à développer au mois de Mai. Les développements au sein de PPIL se font en fonction de l'évolution du sprint en cours. Je suis arrivé dans un contexte de corrections d'anomalies. C'est naturellement que des (QC) corrections d'anomalies m'ont été attribuées.

Ma première tâche a été de normaliser la charte graphique en même temps que l'interface du projet dans le but d'harmoniser les deux. Cette tâche fait suite à un retour du client (FT). Cela m'a permis de découvrir l'environnement technique. 

Ensuite, j'ai principalement corrigé des anomalies et développé quelques évolutions.

\subsection{Corrections d'anomalies}

Lorsque les BA détectent une anomalie, ils l'identifient et la répertorie dans l'outil HPALM. Il est important de préciser que certaines anomalies sont détectées par le client. Elles sont classées par priorité et importance.
Les FT (anomalie retour client) sont souvent prioritaires par rapport aux anomalies détectées par l'équipe.
Un lot correspond à une version de l'application, chaque anomalie est rattaché à un lot.

\begin{figure}[h]
\centering
\includegraphics[width=0.8\textwidth]{images/HPLMliste.png}
\caption{HPALM : Anomalies corrigées de la release 20.06.00}
\end{figure}

\subsection{Démarche Adoptée}

Avant de commencer le développement, j'ai dû lire les consignes de code, règles à respecter soigneusement car mes développements seront livrés directement.

J'ai été chargé de corriger les anomalies distribuées par les RT.
Le référent technique a attribué différentes QC (Anomalies) aux différents SB. 
Les anomalies que j'ai corrigées étaient toutes différentes et incluaient différentes technologies à chaque fois.

Quelques exemples d'anomalies que j'ai corrigées : 
\begin{itemize}
    \item Suppression d'un élément qui ne se supprime pas en base ;
    \item Erreur dans le chargement d'une page ;
    \item Bouton "annuler" non présent ;
    \item Ordre ou classement incorrect ;
    \item Re-direction de page incorrecte lors de la consultation...
\end{itemize}

Pour mener à bien ces différentes corrections, il a été important de prendre du recul avant chaque correction. De bien identifier le périmètre de l'anomalie afin d'éviter les régressions.

Tout d'abord, j'ai du comprendre et analyser l'architecture du programme afin d'identifier plus facilement la provenance des anomalies. 

Chacun des points suivants ont été très importants dans ma démarche de correction d'anomalies :
\begin{itemize}
    \item Étudier l'architecture de la base de données a été primordial pour avoir une visibilité sur les relations entre les différentes entités ;
    \item Me former sur les technologies ;
    \item Comprendre l'anomalie aussi bien fonctionnement que techniquement ;
    \item Analyser l'ampleur et le périmètre de l'anomalie :
    \begin{enumerate}
        \item Pour quel type d'utilisateur ?
        \item Dans quel mode (consultation, restitution) ?
        \item Dans quelles rubrique ( mes projets, portefeuille) ?
    \end{enumerate}
    \item Identifier la provenance du problème ;
    \item Comprendre la logique de développement, pourquoi le projet a été codé de cette manière ?
    \begin{enumerate}
        \item Aller voir les développeurs ;
        \item Aller voir le référent technique ;
        \item Consulter les spécifications techniques ou fonctionnelles.
    \end{enumerate}
    \item Identifier les solutions possibles ;
    \item Choisir la meilleure façon de solutionner le problème :
    \begin{enumerate}
        \item Modifier au minimum le code ;
        \item Trouver la solution la plus optimale ;
        \item Éviter les régressions.
    \end{enumerate}
 \end{itemize}              
            
\subsubsection{Respect des délais (RAE)}

Pour chacune de mes tâches j'ai dû estimer le temps prévu à la réalisation de celle-ci. Respecter les délais est très important sur plusieurs points :
\begin{itemize}
    \item Savoir quand on pourra proposer une version du projet au client.
    \item Savoir quelle tâche est assignée à quel développeur. 
    \item Savoir quelle choix faire pour résoudre une tâche.
\end{itemize}

\subsubsection{Processus et outils utilisés :}

Après avoir développé une évolution, les étapes à suivre sont :
\begin{itemize}
    \item "Pusher" ma branche avec Git sur le dépot distant ;
    \item Créer une "pull request" pour alerter le référent technique ;
    \item Commenter la correction ou l'évolution sur HPALM.
\end{itemize}

\begin{figure}[!h]
\centering
\includegraphics[width=1\textwidth]{images/PullRequest.png}
\caption{Visual Studio Team Foundation : Les Pull Request}
\end{figure}

Ci-dssous le workflow des anomalies dans HPALM :
\begin{figure}[!h]
\centering
\includegraphics[width=0.6\textwidth]{images/DefectHPALM.png}
\caption{HPALM : Defect Workflow}
\end{figure}

\subsection{Quelques exemples d'anomalies corrigées}

Toutes les anomalies ont été intéressantes à corriger. Elles m'ont fait progresser et travailler sur des technologies diverses. J'ai réussi à acquérir les compétences techniques et nécessaires au projet. L'enjeu de la correction de ces anomalies : aucun retour bloquant en qualification interne et externe tout en garantissant la non régression.

\subsubsection{Anomalie [Portefeuille][liste déroulante]}

Visuel de l'anomalie dans HPALM :
\begin{figure}[!h]
\centering
\includegraphics[width=1\textwidth]{images/QC2120.PNG}
\caption{HPALM : Defect 2112}
\end{figure}

On peut voir que l'anomalie est de priorité "urgente" et de gravité "majeure". Elle a été détectée par le client (priorité supplémentaire).

\textbf{Description de la correction à effectuer :} 

[Profil CP ou Manager][portefeuille]
Dans la liste déroulante qui apparaît lorsqu'on veut ajouter des projets ou des lots dans les différents portefeuilles de : 
\begin{itemize}
    \item lots statiques,
    \item lots dynamiques,
    \item projets statiques,
    \item projets dynamiques.
\end{itemize}

Sélectionner par défaut un portefeuille s'il n'y en a qu'un seul. Ne rien sélectionner par défaut sinon. 
[Profil Responsable DSI][portefeuille], Sélectionner par défaut "Mes projets favoris" quel que soit le nombre de portefeuilles existants.

Avant de commencer la correction du code de cette anomalie, j'ai alerté le référent technique : la description de l'anomalie était sujette à interprétation. Le RT a remonté l'information au client, qui a fait un retour en précisant l'anomalie.

Lors de la correction en elle-même, j'ai identifié où faire les modifications de code. Puis au cours de celle-ci, j'ai pu monter en compétence sur du C\#, du Type Script ainsi que le framework Kendo. J'ai pris en compte les différentes conditions pour réaliser les différentes actions correctrices. 

\begin{figure}[!h]
\centering
\includegraphics[width=1\textwidth]{images/correction1.PNG}
\caption{HPALM : Defect 2112}
\end{figure}

\subsubsection{Anomalie [Suppression des lots non effective dans plusieurs cas]}

\textbf{Description de la correction à effectuer :} 
La suppression des lots est non effective selon certains critères :
\begin{itemize}
    \item Un lot ne se supprime pas quand il est dans un portefeuille (le lot se supprime visuellement mais pas en base de donnée, quand on refresh la page il réapparaît) ;
    \item Un lot ne se supprime pas quand il a des demandes rattachées ;
    \item Un lot ne se supprime pas quand il n'a pas de projet de référence.
\end{itemize}

Cette anomalie m'a pris plusieurs jours de correction, les points importants à souligner sont : 
\begin{itemize}
    \item Il a fallu mettre en place plusieurs jeux de données pour visualiser l'anomalie ;
    \item Pendant la correction, j'ai détecté une autre anomalie : un lot sans projet de référence ne s'affiche pas dans un portefeuille (Anomalie que j'ai corrigée par la suite) ;
    \item J'ai proposé une solution efficace et correcte ;
    \item Je suis allé voir Nicolas (le RT du projet), afin d'avoir son retour sur ma correction ;
    \item Il m'a apporté son recul et son expérience. Il m'a proposé de refaire une partie de cette correction en modifiant une procédure stockée en SQL plutôt que de modifier une partie du code. Cette solution étant préférable pour la maintenir le code et l'optimiser.
\end{itemize}

\subsubsection{Anomalie [Onglet synthèse]}

\textbf{Description de l'anomalie :} En restitution d'un projet, onglet synthèse, quand on remonte sur une semaine précédente, les dates ne sont pas les bonnes. 

\begin{figure}[!h]
\centering
\includegraphics[width=1\textwidth]{images/ppil-indicateur-synthese-censored.png}
\caption{Indicateur Synthèse}
\end{figure}

Les difficultés rencontrées lors de cette correction ont été :
\begin{itemize}
    \item Comprendre le fonctionnel, c'est-à-dire la logique des jalons.
    \item Travailler et modifier des requêtes complexes.
    \item Ne pas faire de régression.
\end{itemize}


\subsubsection{L'indicateur Capacity Planning : des expressions multidimensionnelles}

Le Capacity Planning est sans doute l'indicateur le plus complexe de PPIL. En effet, le Capacity Planning récupère les données grâce à des expressions multidimensionnelles (MDX) (Multidimensional Expressions), un langage de requête pour les bases de données OLAP. Je dois donc manipuler des expressions multidimensionnelles afin d'extraire des données stockées dans le Cube. Aucun membre de l'équipe est expert dans cette technologie. J'ai donc adoptée une démarche d'analyse du fonctionnement des requêtes/expressions existantes et montée en compétence sur les bases de données OLAP. Pour le moment, j'ai réussi à corriger une des trois anomalies qui m'ont été confiés. Vous pouvez trouver en annexes une image du Capacity Planning.

%%% Local Variables: 
%%% mode: latex
%%% TeX-master: "isae-report-template"
%%% End: 
\include{04-conclusion}

\appendix
\chapter{Correction Anomalie Capacity Planning}

\begin{figure}[H]
\centering
\includegraphics[width=1\textwidth]{images/capacity-planning-globales-jour.png}
\caption{La correction}
\end{figure}

\begin{figure}[H]
\centering
\includegraphics[width=1\textwidth]{images/excel-capacityplanning.png}
\caption{Test des données générées par ma solution}
\end{figure}

\begin{figure}[H]
\centering
\includegraphics[width=1\textwidth]{images/env-dev.png}
\caption{Environnement de développement}
\end{figure}
\bibliographystyle{authoryear-fr}
\bibliography{references}

https://www.soprasteria.com/fr

https://assurance-maladie.ameli.fr/qui-sommes-nous/notre-fonctionnement/organisation

SFG, SFD, STD de PPIL

Manuel Utilisateur de PPIL

https://docs.microsoft.com/en-us/aspnet/mvc/overview/getting-started/introduction/adding-a-controller

https://www.telerik.com/documentation

https://www.c-sharpcorner.com/

https://www.jhipster.tech/

%\clearpage

%%%%%%%%%%%%%%%%
%%% Abstract %%%
%%%%%%%%%%%%%%%%

%\thispagestyle{empty}
%
%\vspace*{\fill}
%\noindent\rule[2pt]{\textwidth}{0.5pt}\\
%{\textbf{Résumé ---}}
%Lorem ipsum dolor sit amet, consectetur adipiscing elit. Sed non risus. Suspendisse lectus tortor, dignissim sit amet, adipiscing nec, ultricies sed, dolor. Cras elementum ultrices diam. Maecenas ligula massa, varius a, semper congue, euismod non, mi. Proin porttitor, orci nec nonummy molestie, enim est eleifend mi, non fermentum diam nisl sit amet erat. Duis semper. Duis arcu massa, scelerisque vitae, consequat in, pretium a, enim. Pellentesque congue. Ut in risus volutpat libero pharetra tempor. Cras vestibulum bibendum augue. Praesent egestas leo in pede. Praesent blandit odio eu enim. Pellentesque sed dui ut augue blandit sodales. Vestibulum ante ipsum primis in faucibus orci luctus et ultrices posuere cubilia Curae; Aliquam nibh. Mauris ac mauris sed pede pellentesque fermentum. Maecenas adipiscing ante non diam sodales hendrerit. Ut velit mauris, egestas sed, gravida nec, ornare ut, mi. Aenean ut orci vel massa suscipit pulvinar. Nulla sollicitudin. Fusce varius, ligula non tempus aliquam, nunc turpis ullamcorper nibh, in tempus sapien eros vitae ligula. Pellentesque rhoncus nunc et augue. Integer id felis.

%{\textbf{Mots clés :}}
%Lorem ipsum dolor sit amet, consectetur adipiscing elit. Sed non risus. Suspendisse lectus tortor.
\\
\noindent\rule[2pt]{\textwidth}{0.5pt}
\begin{center}
  MIAGE\\
  Université Paris Nanterre\\
  200 avenue de la République\\
  92001 Nanterre Cedex
\end{center}
\vspace*{\fill}

\end{document}