\chapter{Ma mission au sein de PPIL}
\label{sec:unchapitre}
\section{[Objectifs de mission] Intégrer l'équipe PPIL et participer au développement du projet}
Ma mission au cours de ces deux premiers mois de stage a été d'intégrer l'équipe PPIL et de participer au développement du projet. Mes objectifs de mission s'articule autour de trois points :
\subsection{Développement :} 
\begin{itemize}
    \item Acquérir les compétences techniques nécessaires au projet.
    \item Corriger les anomalies affectées par le référent technique sur le périmètre PPIL 
    \item Aucun retour bloquant en qualification interne et externe 
    \item Garantir la non régression
\end{itemize}
\subsection{Qualification : } 
\begin{itemize}
    \item Participer à l'exécution des tests internes : rigueur dans l'exécution des tests.
    \item Remonter les anomalies détectées. Qualifier correctement une anomalie (intitulé, criticité, priorité, type, module, ...) de manière à rendre le plus compréhensible possible l'anomalie, pour le développeur.
\end{itemize}
\subsection{Reporting et Autonomie :} 
\begin{itemize}
    \item Estimation et Respect des délais et des charges
    \item --> Estimer ses charges, suivre son RAE, et le cas échéant expliquer les dérives 
    \item Reporting :
    \item --> Assurer le reporting auprès de son tuteur et remonter les difficultés rencontrées
    \item Capitaliser : 
    \item --> Capitaliser ses travaux sur le wiki et le groupe réseau
\end{itemize}
\section{Fonctionnement de l'équipe}
\subsection{Les différents rôles}
J'ai intégré l'équipe de développement du Portail de PILotage (PPIL) utilisé par les chefs de projet (et autres profils) de la Cnam. L'équipe est composée de 12 collaborateurs. Elle est notamment constituée de :
\begin{itemize}
    \item un CP (chef de projet) 
    \item un RT (référent technique)
    \item un RF (référent fonctionnel)
    \item des BA (business analyst)
    \item des SB (solution builder)
\end{itemize}
Le RT (référent technique) sont les référents fonctionnels, ils gèrent la répartition et l'avancement des tâches de chacun (RAE), il est en contact permanent avec les dév et le CP. Le RF est lui en contact direct avec le client, les BA et le CP.
\subsection{Les processus mis en place}
\subsubsection{processus de livraison}
Les SB (développeurs) développent sur l'environnement de développement.
Les BA disposent de deux environnements de qualification, ils peuvent donc réaliser des tests sur deux versions de PPIL.
A la fin d'un sprint, lorsque les développements évolutions du lot sont terminées, le lot est livré lors d'une "livraison interne" dans l'environnement "qual" de qualification ou les BA peuvent tester la nouvelle version de PPIL.
livraison client = le client teste de son coté avec son équipe;
MEP A la fin d'un lot, mise en production signifie la livraison
sprint de 3 semaines
\subsubsection{processus au sein de l'équipe}
daily meeting tout les jours, 
V1 PPIL, réunions hebdomadaires qui rassemble toute l'équipe PPIL.
\begin{itemize}
    \item point sur l'Agence 151 / plateau (le plateau contient une partie de l'agence), des autres projets de l'agence
    \item point sur PPIL
    \item rétrospective de la semaine, santé, tendance, satisfaction du client
    \item évolutions du projet, attribution des taches
\end{itemize}
%%% Local Variables: 
%%% mode: latex
%%% TeX-master: "isae-report-template"
%%% End: 