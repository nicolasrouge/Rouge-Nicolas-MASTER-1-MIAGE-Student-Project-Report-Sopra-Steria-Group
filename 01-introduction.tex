\chapter*{Introduction}
\addcontentsline{toc}{chapter}{Introduction}
\markboth{Introduction}{Introduction}
\label{chap:introduction}
%\minitoc

Sopra Steria Group m'a accueilli pour mon stage que j'ai effectué dans le cadre de ma formation MASTER 1 MIAGE. Je tenais à effectuer mon stage dans une entreprise novatrice et orientée vers le digital, ce stage a été pour moi l'occasion d'intégrer la vie active dans le domaine qui me motive. Cela a été ma première expérience de développeur en entreprise.

Lors de ma recherche de stage, je me suis naturellement tourné vers Sopra Steria, entreprise de service numérique qui est (un des) leader européen de la transformation numérique.
J'ai postulé chez Sopra Steria pour la diversité des projets et des secteurs d'activités ainsi que le grand nombre de métiers. Différentes qualités que l'on retrouve chez les ESN. 

Sopra Steria Group est née de la fusion de deux SSII : Sopra et Steria en Janvier 2015.
L'entreprise intervient sur plusieurs secteurs, chaque secteur est constitué de plusieurs agences qui fonctionnent comme des entreprises indépendantes. 

C'est à la suite d’un entretien que j'ai été recruté au sein de l’une des 4 agences s'occupant du secteur public : l’agence 151 qui s’occupe du domaine "Santé, Social, Emploi".
Chaque agence est divisée en plusieurs équipes qui gèrent différents projets.

Lors de ce stage, mon objectif était de découvrir le métier de développeur au sein d'une entreprise. Je suis heureux d'avoir pu intégrer une équipe. j'ai pu découvrir tout les aspects de gestion d'un projet, avec les différentes étapes, procédures et des différents rôles que chacun occupe.

J'ai intégré l'équipe PPIL qui développe un outil pour la CNAM. PPIL est un portail de pilotage qui permet aux différents acteurs de la CNAM. L'outil existe depuis 2010. Il a pour objectif, de permettre en fonctions de leurs profils (par exemple Manager ou Chef de projet) d'effectuer le reporting et le suivi des différents projets. 

Il est primordial de préciser du client du projet qu'est la CNAM, gros organisme du secteur public français. (tres developpé et unique au monde) L'organisme est en pleine révolution digitale. Sopra Steria Group et ses collaborateurs l’accompagnent sur le chemin de la modernisation.
C’est dans ce contexte que je suis intervenu sur PPIL. 

Travailler pour un client du service public a un intérêt particulier pour moi car les actions menées au quotidiens par Sopra Steria ont un impact direct sur la vie de nos concitoyens.
Dans le cas du projet PPIL, l'impact est d'abord de faciliter le fonctionnement interne de la CNAM et de leur permettre une optimisation de gestion et de reporting des leurs projets. qui a pour but le bon déroulement de leurs projets qui eux ont un impact direct sur nos concitoyens.

Je suis aussi intervenu sur un projet "stagiaires" interne à Sopra Steria Pôle métier agence 151 : Gestion des compétences des différents intervenants du plateau. L'objectif principal de l'outil est de permettre aux chefs de projets ou administrateurs de projets de construire des équipes en fonction des compétences techniques des membres.

Lors de ces 4 mois de stage, j'ai pu intervenir sur le projet PPIL, et sur un projet stagiaire interne à l'agence. Cela m'a permis de développer mes connaissances et compétences.

%%% Local Variables: 
%%% mode: latex
%%% TeX-master: "isae-report-template"
%%% End: 
