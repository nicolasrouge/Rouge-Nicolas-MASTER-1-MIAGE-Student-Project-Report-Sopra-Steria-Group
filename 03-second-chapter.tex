\chapter{Mon rôle au sein de PPIL}
\label{sec:unchapitre}

Au cours de mon stage, j'ai participé aux différentes phases de réalisation d'un projet : 
\begin{itemize}
    \item \textbf{phases de développement et qualification} avec le projet PPIL.
    \item \textbf{phases de conception et de cadrage} avec le projet MATRIX.
    \item \textbf{phase de relation client}, projet MATRIX.
\end{itemize}

\section{Intégrer l'équipe PPIL en participant au développement et à la qualification du projet}

Ma mission a été d'intégrer l'équipe PPIL et de participer au développement et à la qualification du projet en occupant les rôles de SB et BA. Mes objectifs de mission s'articulent autour de trois points :

\subsection{La qualification} 

Les objectifs en terme de qualification sont :
\begin{itemize}
    \item Participer à l'exécution des tests internes avec rigueur ;
    \item Remonter les anomalies détectées ;
    \item Rédiger des plans de tests ;
    \item Qualifier correctement une anomalie de manière à rendre le plus compréhensible possible l'anomalie ;
    \item Prendre en compte les retours client.
\end{itemize}

\subsection{Le développement} 

Lors du développement, je dois :

\begin{itemize}
    \item Acquérir les compétences techniques nécessaires au projet ; 
    \item Corriger les anomalies affectées par le référent technique sur le périmètre PPIL ;
    \item Garantir aucun retour bloquant en qualification interne et externe ;
    \item Garantir la non régression.
\end{itemize}

\subsection{Reporting et Autonomie}

Lors de mon stage, il est important d'être autonome dans la réalisation des tâches notamment en estimant et en respectant les délais. Il faut aussi bien s'intégrer dans l'équipe ainsi que savoir remonter les bonnes informations aux bonnes personnes. Les points que j'ai respectés sont :
\begin{itemize}
    \item Estimer ses charges, suivre son RAE, et le cas échéant expliquer les dérives ;
    \item Assurer le reporting auprès de son tuteur et remonter les difficultés rencontrées ;
\end{itemize}

\section{Fonctionnement de l'équipe}

\subsection{Les différents rôles}

J'ai intégré l'équipe de développement du Portail de PILotage (PPIL) utilisé par les chefs de projet (et autres profils) de la CNAM. L'équipe est composée de 12 collaborateurs : un chef de projet, deux référents, des développeurs (SB) et des analystes fonctionnels (BA).
Le référent technique gère la répartition et l'avancement des tâches de chacun (RAE), il est en contact permanent avec les développeurs et le chef de projet. Le référent fonctionnel est lui en contact direct avec le client, les business analystes et le chef de projet.

\subsection{Les processus mis en place}
\subsubsection{Les différents environnements}
Les BA disposent de deux environnements :
\begin{itemize}
    \item L'environnement de qualification
    \item L'environnement de test
\end{itemize}
Ils peuvent donc réaliser des tests sur deux versions différentes de PPIL simultanément. Les SB disposent de leurs propre environnement de développement. J'ai eu l'occasion de travailler sur tous ces environnements.
\subsubsection{Les processus de livraison}
A la fin d'un sprint, lorsque les développements sont terminés, le lot est livré lors d'une "livraison interne" dans l'environnement "qual" de qualification ou les BA peuvent tester la nouvelle version de PPIL. Après avoir été testé en interne, le lot est livré au client. Le client peut alors tester de son coté avec son équipe. L'étape finale est la mise en production (MEP) qui signifie que l'outil est  déployé chez le client. Les sprints durent généralement 3 semaines.
\subsubsection{Les processus au sein de l'équipe}
Au sein de l'équipe, nous nous réunissons tous les jours lors des daily-meetings, réunions de courte durée, qui permettent à chacun des membres de s'exprimer sur son avancement ou ses difficultés. Dans le cadre d'une réunion appellé V1, l'équipe se voit une fois par semaine le vendredi après-midi pendant une heure. 
Le "V1 PPIL" est constitué de :
\begin{itemize}
    \item Un point sur l'Agence 151 / plateau (le plateau contient une partie de l'agence);
    \item Un point sur PPIL ; 
    \item Une rétrospective de la semaine, santé du projet, tendance du projet, satisfaction du client ;
    \item L'évolution du projet et attribution des tâches.
\end{itemize}
%%% Local Variables: 
%%% mode: latex
%%% TeX-master: "isae-report-template"
%%% End: 